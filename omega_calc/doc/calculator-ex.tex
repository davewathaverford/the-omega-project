\begin{figure}
{

\begin{verbatim}
# R := { [i] -> [i'] : 1 <= i,i' <= 10 && i' = i+1 };
# R;
{[i] -> [i+1] : 1 <= i <= 9}

# inverse R;
{[i] -> [i-1] : 2 <= i <= 10}
# domain R;
{[i]: 1 <= i <= 9}
# range R;
{[i]: 2 <= i <= 10}

# R compose R;
{[i] -> [i+2] : 1 <= i <= 8}

# R+;
{[i] -> [i'] : 1 <= i < i' <= 10}
#              # closure of R = R union (R compose R) union (R compose R ...

# complement R;
{[i] -> [i'] : i <= 0} union
 {[i] -> [i'] : 10 <= i} union
 {[i] -> [i'] : 1 <= i <= 9, i'-2} union
 {[i] -> [i'] : 1, i' <= i <= 9}

# S := {[i] : 5 <= i <= 25};
# S;
{[i]: 5 <= i <= 25}

# R(S);
{[i]: 6 <= i <= 10}
#            # apply R to S

# R \ S;
{[i] -> [i+1] : 5 <= i <= 9}
#           # restrict domain of R to S

# R / S;
{[i] -> [i+1] : 4 <= i <= 9}
#           # restrict range of R to S

# (R\S) union (R/S);
{[i] -> [i+1] : 4 <= i <= 9}
# (R\S) intersection (R/S);
{[i] -> [i+1] : 5 <= i <= 9}
# (R/S) - (R\S);
{[4] -> [5] }

# S*S;
{[i] -> [i'] : 5 <= i <= 25 && 5 <= i' <= 25}
#             # cross product 
\end{verbatim}
}
\label{figure:example1}
\caption{Example of the Omega Calculator in action: part 1}
\end{figure}




\begin{figure}
{

\begin{verbatim}
# D := S - {[9:16:2]} - {[17:19]};
# D;
{[i]: 5 <= i <= 8} union
 {[i]: Exists ( alpha : 2alpha = i && 10 <= i <= 16)} union
 {[i]: 20 <= i <= 25}

# T :=  { [i] : 1 <= i <= 11 & exists (a : i = 2a) };
# T;
{[i]: Exists ( alpha : 2alpha = i && 2 <= i <= 10)}

# Hull T;

{[i]: 2 <= i <= 10}

# Hull D;

{[i]: 5 <= i <= 25}

# codegen D;
for(t1 = 5; t1 <= 8; t1++) {
  s1(t1);
  }
for(t1 = 10; t1 <= 16; t1 += 2) {
  s1(t1);
  }
for(t1 = 20; t1 <= 25; t1++) {
  s1(t1);
  }

# codegen {[i,j] : 1 <= i+j,j <= 10};
for(t1 = -9; t1 <= 9; t1++) {
  for(t2 = max(-t1+1,1); t2 <= min(-t1+10,10); t2++) {
    s1(t1,t2);
    }
  }
\end{verbatim}
}
\label{figure:example2}
\caption{Example of the Omega Calculator in action: part 2}
\end{figure}
