\documentstyle[fullpage]{article}

%
% tex commands used in Omega test papers
%
\def\ft{\footnotesize}
\def\lb{{\{$\ast$}}
\def\rb{{$\ast$\}}}
\def\subE{\stackrel{{\em sub}}{=}}
\def\True{{\rm True}}
\def\gist{{\rm gist\; }}
\def\given{{\rm \;given\; }}
\def\st{{\rm \ s.t.\ }}
\def\t{\qquad}

%
% loop index vectors
\def\i{{\cal I}}
\def\j{{\cal I'}}
\def\k{{\cal I''}}
\def\h{{\cal I'''}}
%
% symbolic constants
\def\Sym{{\tt Sym}}
\def\l{{\tt l}}
\def\u{{\tt u}}
\def\n{{\tt n}}
\def\m{{\tt m}}
\def\p{{\tt p}}
\def\q{{\tt q}}
\def\x{{\tt x}}
\def\y{{\tt y}}
% dont use \c - it screws up \c in bibliography
\def\const{{\rm c}}
%
% define values of loop variables i and j (and path)
% as they appear as part of the vectors i, j, and k
%
% for historical reasons, the 1st, 2nd, 3rd occurances of i are
% written \ii, \ji, \ki   :-P
%
\def\ii{{\rm i}}
\def\ji{{\rm i'}}
\def\ki{{\rm i''}}
\def\hi{{\rm i'''}}
\def\ij{{\rm j}}
\def\jj{{\rm j'}}
\def\kj{{\rm j''}}
\def\hj{{\rm j'''}}
\def\ik{{\rm k}}
\def\jk{{\rm k'}}
\def\kk{{\rm k''}}
\def\hk{{\rm k'''}}
\def\ipath{{\cal P}}
\def\jpath{{\cal P'}}
\def\kpath{{\cal P''}}
%
% wildcards
\def\wilda{{\alpha}}
\def\wildb{{\beta}}
\def\wildc{{\gamma}}
\def\wildd{{\delta}}
%
% floor and ceiling constraint wildcards
\def\wildfloor{{\rm f}}
\def\wildceil{{\rm c}}
%
% loop variables used in examples, not code
\def\loopexa{{\tt r}}
\def\loopexb{{\tt r'}}
%
% Boolean values used in text:
\def\true{{\bf True}}
\def\false{{\bf False}}
%
\def \overC #1 {\lfloor #1 \rfloor}
\def \underC #1 {\lceil #1 \rceil}
%
% introduce a new term
\newcommand{\newterm}[1]{{\em #1}}
%
% use a language keyword like ``for'' or ``if''
\newcommand{\keyword}[1]{{\tt\bf #1}}
%
% refer to code in text
\newcommand{\code}[1]{{\tt #1}}
%
% allow the definition of example numbers
% e.g.  The first time, use ``\labelExample{foo}''
% then later just use \ref{example:foo}
%
\newcounter{eXample}
\newcommand{\labelExample}[1]{\refstepcounter{eXample} \label{eXample:#1} \theeXample}
%
% Make a note about something to be fixed
%
\def \fixMe #1 {{\large\bf #1 \marginpar{FIX}
        \typeout{**********************************************}
        \typeout{FIX ME: #1}
        \typeout{**********************************************}
        }}

\newcommand{\blist}{\begin{list}{$\bullet$}{\setlength{\parsep}{0.1ex}\setlength{\itemsep}{0.2ex}\setlength{\topsep}{0.1ex}}}
\newcommand{\blistA}{\begin{list}{$\bullet$}{\setlength{\itemsep}{0.5ex}\setlength{\topsep}{0.5ex}}}
\newcommand{\blistB}{\begin{list}{$\bullet$}{\setlength{\parsep}{0.2ex}\setlength{\itemsep}{0.3ex}\setlength{\topsep}{0.3ex}}}
%\newcommand{\blistC}{\begin{list}{--}{\setlength{\parsep}{0ex}\setlength{\itemsep}{0.1ex}\setlength{\topsep}{0.3ex}}}
\newcommand{\elist}{\end{list}}

\newcommand{\var}[1]{{\em #1}}
\newcommand{\fun}[1]{{\sf #1}}
\newcommand{\word}[1]{\verb"#1"}
\newcommand{\comment}[1]{{\rm $(*$ #1 $*)$}}



\begin{document}
\bibliographystyle{alpha}
\title{{\bf The Omega Calculator and Library}, {\em version 1.1.0}}


\author{
Wayne Kelly, Vadim Maslov, William Pugh, \\
Evan Rosser, Tatiana Shpeisman, Dave Wonnacott
}

\maketitle


\section{Introduction}

This document gives an overview of the Omega library and 
describes the Omega Calculator, a text-based interface to
the Omega library. A separate document describes the C++ interface 
to the Omega library and we are still working on a third document that describes
some of the algorithms used in implementing the Omega library.

The Omega library 
manipulates integer tuple relations and sets, such as
$$\{[i,j] \rightarrow [j,j'] : 1 \leq i < j < j' \leq n\}
{\rm\ and\ } \{ [ i ,j ] : 1 \leq i < j \leq n\}$$
Tuple relations and
sets are described using Presburger 
formulas\cite{presburger1,Shostak77,presburger2,presburger2a} a class of logical
formulas which can be built from affine constraints over integer
variables, the logical connectives $\neg$, $\land$ and $\lor$, and the
quantifiers $\forall$ and $\exists$.
The best known upper bound on the performance of an algorithm for
verifying Presburger formulas is $2^{2^{2^{n}}}$ \cite{presburger3},
and we have no
reason to believe that our method provides better worst-case
performance. However,  we have found it
to be reasonably efficient for our applications.

The following relation, which maps 2-tuples to 1-tuples:
$\{[i, j] \rightarrow [i] : 1 \le i,j \le 2\}$
represents the following set of mappings: $\{[1,1]\rightarrow[1],\
[1,2]\rightarrow[1],\ [2,1]\rightarrow[2],\ [2,2]\rightarrow[2]\}$.
In addition to variables in the input and output tuples, the Presburger 
formulas may also contain free variables. This allows parameterized relations to be described. 
For example, $n$ and $m$ are free in 
$\{ [i, j] \rightarrow [i] : 1 \le i \le n \land 1 \le j \le m \}$. 
The language allows new relations to be defined in terms of existing relations 
by providing a number of relational operators.  The relational operations 
provided include intersection, union, composition, inverse, domain, range and complement.  
For example, $\{[p] \rightarrow [2p, q]\}$ {\tt compose}
$\{[i,j]\rightarrow[i]\}$ evaluates to
$\{[i,j]\rightarrow[2i,q]\}$.

Relations are simplified before being displayed to the user.  This
involves transforming them into disjunctive normal form and removing
redundant constraints and conjuncts.  During simplification, it may be determined
that the relation contains no points or contains all points, in which
case the simplified constraints will be False or True respectively.
For example, $\{[i]\rightarrow[i]\}$ {\tt intersect}
$\{[i]\rightarrow[i+1]\}$ evaluates to $\{[i]\rightarrow[i'] : {\tt
False}\}$.

The copyright notice and legal fine print for the Omega calculator and library are contained in
the README and omega.h files. Basically, you can do anything you want with
them (other than sue us); if you redistribute it
you must include a copy of our copyright notice and legal fine print.

\section{Omega Calculator invocation, syntax and semantics}

The calculator reads from the file given as the first argument, or
standard input, and prints results on standard output.  
You can specify several command line flags. Using {\tt -D}{\em mk},
where $m$ is a character and $k$ is a digit, sets the debugging level
to $k$ for module $m$. Using {\tt -G}$g$ sets the maximum number
of inequalities in a conjunct to $g$. Using {\tt -E}$e$ sets the maximum number
of equalities in a conjunct to $E$. In version 1.1.0, we intend to 
change our data structures so that these will not need to be specified.
There is also a limit on the maximum number of variables in a conjunct,
but this cannot be changed at run-time. It is given by
{\tt maxVars} in {\tt oc.h}. We also intend to make this go away.


The input is a
series of statements terminated by semicolons. A \# indicates that
the rest of the line is a comment.  
The syntax {\tt <<}{\em filename}{\tt >>} can occur anywhere
(and indicates that the text of the file is to be included here.
The statements can have one of the
forms listed in Figure~\ref{IPIstatements}.  {\em RelExpr} is a short
form of {\em Relational Expression}.

\begin{figure}[tbh]
\begin{tabular}{l|p{4.5in}}
Syntax & Semantics \\\hline

{\tt symbolic} {\em varList} & 
  Defines the variable names as symbolic
  constants that can be used in all later
  expressions. \\

{\em var} := {\em RelExpr} & 
  Computes the relational expression and binds the
  variable name to the result. \\

{\em RelExpr} & 
  Computes and prints the relational expression. 	\\

{\tt codegen} $\ldots$ &
	This is described in Section \ref{section:codegen}.
	\\

{\em RelExpr} {\tt subset} {\em RelExpr} &
  Prints True if the first relation is a subset of the second,
  otherwise prints False. \\
\end{tabular}
\caption{Omega Calculator statements}
\label{IPIstatements}
\end{figure}


Figures \ref{figure:example1} and \ref{figure:example2} show a sample
session with the Omega
Calculator. Lines starting with \# are input to the Omega
calculator, the other lines are output from the calculator.

\input{calculator-ex.tex}

Some relational operations may not preserve
the names of input and output variables.
If this happens, the variables get default names:
{\tt In\_}$n$ for input variables and
{\tt Out\_}$n$ for output variables,
where $n$ is a position of variables in its tuple.
When printing, primes are added to variables to distinguish
between multiple variables with the same name. 
In input, primes may also be used to distinguish between multiple variables 
with the same name: the primes are stripped before being passed
to the Omega library.

\section{Relations}

\subsection{Building relations}

A relation is an operand of a relational expression.
Its syntax is:
\begin{quote}
{\tt \{ [} {\em InputList} {\tt ] -> [} {\em OutputList} {\tt ] :} 
{\em formula} {\tt \}}
\end{quote}

{\em InputList} and {\em OutputList} are lists of tuple elements.
A tuple element can be:

\begin{tabular}{cp{5in}}
{\em var} & The corresponding tuple variable is given this name.
	If a variable with that name is already in scope,
	an equality is added
	forcing this tuple variable to be equal to the value
	of the previous definition.
\\
{\em exp} & The tuple variable is unnamed and forced to be equal
	to the expression.
\\
{\em exp}:{\em exp} & The tuple variable is unnamed and forced to be
	greater than or equal to the first expression and less than
	or equal to the second.
\\
{\em exp}:{\em exp}:{\em int} & The tuple variable is unnamed and forced to be
	greater than or equal to the first expression and less than
	or equal to the second, and the difference between the tuple
variable and the first expression must be an integer multiple of the
	integer.
\\
{\tt *} &The tuple variable is unnamed.
\end{tabular}


The {\em formula} is optional.  If it is omitted, no constraints other than
those introduced by the input and output expressions are imposed upon
the relation's variables.  Otherwise, the formula describes additional
constraints on variables used in the relation.  

\subsection{Sets}
In addition to relations, the system can represent {\em sets}.

When a relation is declared with only one tuple, as in:
\begin{quote}
{\tt \{ [} {\em SetList} {\tt ] :} {\em formula} {\tt \}}
\end{quote}
then the relation becomes a set.
The variables that are used to describe a set ({\em SetList}) 
are called {\em set variables} rather than input or output variables.

\subsection{Presburger formula operations}
As mentioned above, the tuples belonging to the relation are defined
by a {\em Presburger formula}.  This formula is built from constraints
using the operations described in Figure~\ref{FormulaOps}.

\begin{figure}[tbh]
\begin{tabular}{l|lll}
Name & \multicolumn{3}{l}{Notation} \\\hline

And & {\em formula} {\tt \&} {\em formula} & {\em formula} {\tt \&\&} {\em formula} & {\em formula} {\tt and} {\em formula} \\

Or  & {\em formula} {\tt |} {\em formula}  & {\em formula} {\tt ||} {\em formula}   & {\em formula} {\tt or} {\em formula} \\

Not & {\tt !} {\em formula}  &                    & {\tt not} {\em formula}\\

Exists & \multicolumn{3}{l}{{\tt exists} ($v_1,...,v_n$: {\em formula})} \\

Forall & \multicolumn{3}{l}{{\tt forall} ($v_1,...,v_n$: {\em formula})} \\

Parentheses & ({\em formula}) \\
Constraint & {\em constraint} \\
\end{tabular}
\caption{Presburger formula syntax}
\label{FormulaOps}
\end{figure}


\subsection{Constraints and arithmetic and comparison operations}

A Presburger formula is built from constraints using the operations
described in the previous subsection.  In this subsection we describe
the syntax of individual constraints.

A constraint is a series of expression lists, connected with
the arithmetic relational operators {\tt =, !=, >, >=, <, <=}.
An example is {\tt 2i + 3j <= 5k,p <= 3x-z = t}.

When an constraint contains a comma-separated list of expressions, it
indicates that the same constraints should be introduced for each of
the expressions.  The constraint {\tt a,b <= c,d > e,f} translates to
the constraints
$$a \leq c \land a \leq d \land b \leq c \land b \leq d \land 
c > e \land c > f \land d > e \land d > f$$

Expressions can be of the forms (where {\em var} is a variable, 
{\em integer} is an integer constant, and $e$, $e_1$, and $e_2$ are 
expressions):
{\em var}, {\em integer}, $e$, {\em integer} $e$,
$e_1 + e_2$, $e_1 - e_2$, 
$e_1 \ast e_2$, $- e$, $(e)$.

An important restriction is that all expressions in
the constraints must be affine functions of the variables.
For example, $2\ast x$ is legal, $x \ast 2$ is legal, but $x \ast x$ is
illegal.

\section{Relational and set operations}

A relational expression is an expression over individual relations.
The relational operations defined in the system are listed
in Figures \ref{RelOps1} and \ref{RelOps2}. 
Here $r, r_1, r_2$ are relations and $s, s_1,
s_2$ are sets.


\begin{figure}[tbp]
\begin{tabular}{l|l|p{3.5in}}
Name & Syntax & Explanation \\\hline

Union & 
$r = r_1$ {\tt union} $r_2$ 
& $x \rightarrow y \in  r$ iff $x \rightarrow y \in  r_1$
or $x \rightarrow y \in  r_1$
\\

& 
$s = s_1$ {\tt union} $s_2$ 
& $x \in  s$ iff $x \in  s_1$
or $x \in  s_1$
\\

Intersection & 
$r = r_1$ {\tt intersection} $r_2$
& $x \rightarrow y \in  r$ iff $x \rightarrow y \in  r_1$
and $x \rightarrow y \in  r_1$
\\

& 
$s = s_1$ {\tt intersection} $s_2$
& $x \in  s$ iff $x \in  s_1$
and $x \in  s_1$
\\

Difference &
$r = r_1 - r_2$ 
& $x \rightarrow y \in  r$ iff $x \rightarrow y \in  r_1$
and $x \rightarrow y \not\in  r_1$
\\

&
$s = s_1 - s_2$ 
& $x \in  s$ iff $x \in  s_1$
and $x \not\in  s_1$
\\

Complement &
$r =$ {\tt complement} $r_1$ 
& $x \rightarrow y \in  r$ iff $x \rightarrow y \not\in  r_1$
\\

&
$s =$ {\tt complement} $s_1$ 
& $x \in  s$ iff $x \not\in  r_1$
\\

Composition &
$r = r_1$ {\tt compose} $r_2$ &
$x \rightarrow y \in r$ iff $\exists y \st x \rightarrow y \in r_2$
and $ y \rightarrow z \in r_1$
\\

Application &
$s = r_1 (s_2)$ &
$x \in s$ iff $\exists y \st x \rightarrow y \in r_1$
and $ y \in s_2$
\\


&
$r = r_1 (r_2)$ &
Equivalent to $r_1$ {\tt compose} $r_2$ \\

Join &
$ r = r_1 . r_2 $
& Equivalent to $r_2$ {\tt compose} $r_1$\\
& & $x \rightarrow y \in r$ iff $\exists y \st x \rightarrow y \in r_1$
and $ y \rightarrow z \in r_2$
\\



Inverse &
$r =$ {\tt inverse} $r_1$ &
$x \rightarrow y \in r $ iff $y \rightarrow x \in r_1$\\

Domain &
$s =$ {\tt domain} $r_1$ 
& $x \in s$ iff $\exists y \st x \rightarrow y \in r_1$\\

Range &
$s =$ {\tt range} $r_1$ 
& $y \in s$ iff $\exists x \st x \rightarrow y \in r_1$\\

Restrict Domain &
$r = r_1 ~\verb+\+~ s_2$ 
& $x \rightarrow y \in r$ iff $x \rightarrow y \in r_1$ and $x \in s_2$
\\

Restrict Range &
$r = r_1 ~/~ s_2$ 
& $x \rightarrow y \in r$ iff $x \rightarrow y \in r_1$ and $y \in s_2$
\\

Gist &
$r=$ {\tt gist} $r_1$ {\tt given} $r_2$ &
Computes gist of relation $r_1$ given relation $r_2$. \\


\end{tabular}
\caption{Relational operations, part 1}
\label{RelOps1}
\end{figure}


\begin{figure}[tbp]
\begin{tabular}{l|l|p{3.5in}}
Name & Syntax & Explanation \\\hline

Hull & $r=$ {\tt hull} $r_1$ & Computes an single convex region that
contains all of $r_1$.  Not as tight as the convex hull, but intended
to be fairly fast.  \\

Convex Hull &
$r=$ {\tt ConvexHull} $r_1$ &
Computes the convex hull \cite{Schrijver86,Wilde93} of $r_1$.
May be prohibitively expensive to compute
and/or induce numeric overflow of coefficients (leading
to a core dump)
\\

Affine Hull &
$r=$ {\tt AffineHull} $r_1$ &
Computes the affine hull \cite{Schrijver86,Wilde93} of $r_1$.
May be prohibitively expensive to compute
and/or induce numeric overflow of coefficients (leading
to a core dump)
\\

Linear Hull &
$r=$ {\tt LinearHull} $r_1$ &
Computes the linear hull \cite{Schrijver86,Wilde93} of $r_1$.
May be prohibitively expensive to compute
and/or induce numeric overflow of coefficients (leading
to a core dump)
\\

Conic Hull &
$r=$ {\tt ConicHull} $r_1$ &
Computes the conic or cone hull \cite{Schrijver86,Wilde93} of $r_1$.
May be prohibitively expensive to compute
and/or induce numeric overflow of coefficients (leading
to a core dump)
\\

Farkas Lemma &
$r=$ {\tt farkas} $r_1$ &
Applies Farkas's lemma \cite{Schrijver86,Wilde93} to $r_1$.
In the result, the values of the variables
correspond to legal values for coefficients
of equations implied by the convex hull of $r_1$.
Also happens to represent the convex hull of $r_N$
using a points and rays representation. 	 
See \cite{Wilde93} for a more detailed explaination.
\\

Convex Check &
$r=$ {\tt ConvexCheck} $r_1$ &
Returns the convex hull of $r_1$ if we can easily determine
that it is equal to $r_1$, otherwise returns $r_1$.
\\

Pairwise Convex Check &
$r=$ {\tt PairwiseCheck} $r_1$ &
Checks to see if any two conjuncts in $r_1$ can be replaced
exactly by their convex hull (doing so if possible).
\\

Transitive Closure &
$r = r_1 {\tt +}$ &
Least fixed point of $r^+ \equiv r \cup (r \circ r^+)$
\\

&
$r = r_1 {\tt +}$ {\tt within} $b$ &
Least fixed point of $r^+$ for dependence relation $r$ within iteration space $b$
\\

Conic Closure &
$r = r_1 {\tt @}$ &
Gives a simple dependence relation $r$ such that
any linear schedule for  all of the dependences in 
$r$ is non-negative if and only
if it is legal for all of the dependences in $r_1$.
Note that $r_1 {\tt +} \subseteq r_1 {\tt @}$.
We previously refered to this as affine closure \cite{uniform2.4}.
\\


Cross-product &
$r = r_1$ {\tt *} $r_2$ &
$x \rightarrow y \in r $ iff $x \in r_1$ and $y \in r_2$
\\

Create superset &
$r= $ {\tt supersetof} $r_1$ &
$r$ is inexact with lower bound $r_1$ 
\\

Create subset &
$r= $ {\tt subsetof} $r_1$ &
$r$ is inexact with upper bound $r_1$ 
\\

Create upper bound &
$r= $ {\tt upper\_bound} $r_1$ &
$r$ is an exact relation and is an upper bound on $r_1$ (all UNKNOWN
constraints in $r_1$ are interpreted as TRUE)
\\

Create lower bound &
$r= $ {\tt lower\_bound} $r_1$ &
$r$ is an exact relation and a lower bound on $r_1$ (all UNKNOWN
constraints in $r_1$ are interpreted as FALSE)
\\


Get an example &
$r= $ {\tt example} $r_1$ &
$r \subseteq r_1$ and all variables in $r$ are single-valued
\\

Symbolic example &
$r= $ {\tt sym\_example} $r_1$ &
$r \subseteq r_1$ and all non-symbolic variables in $r$ are single-valued

\end{tabular}
\caption{Relational operations, part 2}
\label{RelOps2}
\end{figure}

\section{Code Generation}
\label{section:codegen}
The Omega Calculator incorporates an algorithm for generating code for
multiple, overlapping iteration spaces \cite{Uniform4}.  Each
iteration space has an associated statement or block of statements.
The syntax is 
\centerline{{\tt codegen} [{\em effort}] $IS_1, IS_2, \ldots ,IS_n$ [ {\tt given} {\em known}] }
Each
iteration space $IS_i$ can be specified either as a set representing
the iteration space, or as an original iteration space and a
transformation, {\em T}:{\em IS}, where {\em IS} is the original
iteration space and {\em T} is a relation defining an affine, 1-1 mapping
to a new iteration space. That is, given a point in the original
iteration space, the mapping specifies the point in the new iteration
space at which to execute that iteration.

The effort value specifies the amount of effort to be used to
eliiminate sources of overhead in the generated code. Sources of
overhead include if statements an {\tt min} and {\tt max} functions
in loop bounds. If not specified, the effort level is 0. The different effort
levels are:
\begin{description}
\item{-2}	Minimal possible effort. Loop bounds may not be finite.
\item{-1}	Forces finite bounds
\item{0}	Forces finite bounds and tries to remove {\tt if}'s 
		within most deeply nested loops (at a possible cost
		of code duplication).
\item{1}	removes {\tt if}'s within most deeply nested loops
		and loops one short of being most deeply nested.
\item{2}	$\ldots$ and loops two short of being most deeply nested.
\item{$x$}	$\ldots$ and loops $x$ short of being most deeply nested.
\end{description}

The known information, if specified, is used to simplify the generated
code. The generated code will not be correct if known is not true.
Currently, the known relation needs to be a set with the same arity and the transformed
iteration space.

A discussion of program transformations using this framework is
given in \cite{Uniform2.1,Uniform2.3,Uniform3}.

The following is an example of code generation, given three original
iteration spaces and mappings.  The transformed code requires the
traditional transformations loop distribution and imperfectly nested
triangular loop interchange.   Below, the program information has been
extracted and presented to the Omega Calculator in relation form.

\vspace{0.2in}

{\bf Original code}
\begin{verbatim}
      do 30 i=2,n
10      sum(i) = 0.
        do 20 j=1,i-1
20        sum(i) = sum(i) + a(j,i)*b(j)
30      b(i) = b(i) - sum(i)
\end{verbatim}

{\bf Schedule} (for parallelism)
$$T10:\{\ [\ i\ ] \rightarrow\   [\ 0,i,0,0\ ]\ \}$$
$$T20:\{\ [\ i,j\ ] \rightarrow\ [\ 1,j,0,i\ ]\ \}$$
$$T30:\{\ [\ i\ ] \rightarrow\   [\ 1,i-1,1,0\ ]\ \}$$

{\bf Omega Calculator output:}
\input{calculator-ex2.tex}

\section{Inexact Relations}

The special constraint UNKNOWN represents one or more additional
constraints that are not known to the Omega Library.  Such constraints
can arise from uses of uninterpreted function symbols or transitive
closure (as described below), or when the user explicitly requests it.
Such relations require conservative treatment from the library (e.g.,
subtracting a conjunct containing UNKNOWN from a relation must
return that relation, since the unknown constraints might make the
conjunct unsatisfiable.)

The {\tt upper\_bound} and {\tt lower\_bound} operations can be used to
produce exact relations from inexact relations.  They are produced by
treating UNKNOWN constraints as TRUE or FALSE, respectively.



\section{Presburger Arithmetic with Uninterpreted Function Symbols}
The Omega Calculator allows certain restricted uses of 
{\em uninterpreted function symbols} in a Presburger formula.
Functions may be declared in the {\tt symbolic} statement as
\begin{quote}
{\tt symbolic} {\em Function} ({\em Arity})
\end{quote}
where {\em Function} is the function name and 
{\em Arity} is its number of arguments.
Functions of arity 0 are symbolic constants.

Functions may only be applied to a prefix of the input or output tuple
of a relation, or a prefix of the tuple of a set.  The function
application may list the names of the argument variables explicitly
({\em not yet supported}), or use the abbreviations {\em Function({\tt
In})}, {\em Function({\tt Out})}, and {\em Function({\tt Set})}, to
describe the application of a function to the appropriate length
prefix of the desired tuple.

Our system relies on the following observation: Consider a formula $F$
that contains references $f(i)$ and $f(j)$, where i and j are free in
$F$.  Let $F'$ be $F$ with $f_{i}$ and $f_{j}$ substituted for $f(i)$
and $f(j)$.  $F$ if satisfiable iff $((i = j) \Rightarrow (f_{i} =
f_{j})) \land F'$ is satisfiable.  For more details, see \cite{Shostak79}.
% Robert
% Shostak's paper ``A Practical Decision Procedure for Arithmetic with
% Function Symbols'' \cit(JACM, April 1979).

Presburger Arithmetic with uninterpreted function symbols is in
general undecidable, so in some circumstances we will have to produce
approximate results (as we do with the transitive closure operation)
\cite{transitiveClosure}. 

The following examples show some legal uses of uninterpreted function
symbols in the Omega Calculator:

\input{calculator-ex3.tex}


\section{Reachability}

Consider a graph where each directed edge is specified as a tuple
relation.  Given a tuple set for each node representing starting
states at each node, the library can compute which nodes of the graph are
reachable from those start states, and the values the tuples can take
on.  

The syntax is:
\begin{verbatim}
reachable ( [list of nodes] ) { [node:startstates] | node_i->node_j:transition] }
\end{verbatim}

For example,

\begin{verbatim}
reachable (a,b,c)
	  { a->b:{[1]->[2]},
	     b->c:{[2]->[3]},
	     a:{[1]}};
\end{verbatim}

The transitions and start states may be any expression that evaluates
to a relation.

You can also compute a tuple set containing the reachable values a t
given node; for example:

\begin{verbatim}
R := reachable of c in (a,b,c)
	                { a->b:{[1]->[2]},
	                  b->c:{[2]->[3]},
	                  a:{[1]}};
\end{verbatim}

The current implementation is very straightforward and can be very slow.

\section{Current limitations}

The transitive closure operation will not work on a relation with
uninterpreted function symbols of arity $> 0$.  Any operation that
requires the projection of input or output variables (such as
composition) may return inexact results if variables in the argument
list of a function symbol are projected.





\bibliography{para}

\end{document}
 

